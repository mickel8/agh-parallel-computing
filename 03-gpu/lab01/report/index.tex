\documentclass{article}
\usepackage[polish]{babel}
\usepackage[backend=biber,style=numeric]{biblatex}
\usepackage{graphicx}
\usepackage[hidelinks]{hyperref}
\usepackage{listings}
\usepackage{amsmath}
\usepackage{tikz}
\usepackage{pgfplots}
\usepackage{cleveref}
\usepackage{float}


%\bibliography{bibliography}

\lstset{
    breaklines=true,
    basicstyle=\ttfamily}

\title{GPU 1}
\author{Michał Śledź}
\date{12 maja 2020}

\begin{document}
    \maketitle

    \section{Środowisko testowe}
    \begin{description}
        \item [procesor]: i5-9600K 6 rdzeni, 3,70-4,60 GHz, 9MB Cache
        \item [pamięć]: 16 GB RAM, 3200 MHz, CL 16
    \end{description}
    \begin{lstlisting}
    ./deviceQuery Starting...

    CUDA Device Query (Runtime API) version (CUDART static linking)

    Detected 1 CUDA Capable device(s)

    Device 0: "GeForce RTX 2060"
    CUDA Driver Version / Runtime Version          10.2 / 10.2
    CUDA Capability Major/Minor version number:    7.5
    Total amount of global memory:                 5931 MBytes (6219563008 bytes)
    (30) Multiprocessors, ( 64) CUDA Cores/MP:     1920 CUDA Cores
    GPU Max Clock rate:                            1770 MHz (1.77 GHz)
    Memory Clock rate:                             7001 Mhz
    Memory Bus Width:                              192-bit
    L2 Cache Size:                                 3145728 bytes
    Maximum Texture Dimension Size (x,y,z)         1D=(131072), 2D=(131072, 65536), 3D=(16384, 16384, 16384)
    Maximum Layered 1D Texture Size, (num) layers  1D=(32768), 2048 layers
    Maximum Layered 2D Texture Size, (num) layers  2D=(32768, 32768), 2048 layers
    Total amount of constant memory:               65536 bytes
    Total amount of shared memory per block:       49152 bytes
    Total number of registers available per block: 65536
    Warp size:                                     32
    Maximum number of threads per multiprocessor:  1024
    Maximum number of threads per block:           1024
    Max dimension size of a thread block (x,y,z): (1024, 1024, 64)
    Max dimension size of a grid size    (x,y,z): (2147483647, 65535, 65535)
    Maximum memory pitch:                          2147483647 bytes
    Texture alignment:                             512 bytes
    Concurrent copy and kernel execution:          Yes with 3 copy engine(s)
    Run time limit on kernels:                     Yes
    Integrated GPU sharing Host Memory:            No
    Support host page-locked memory mapping:       Yes
    Alignment requirement for Surfaces:            Yes
    Device has ECC support:                        Disabled
    Device supports Unified Addressing (UVA):      Yes
    Device supports Compute Preemption:            Yes
    Supports Cooperative Kernel Launch:            Yes
    Supports MultiDevice Co-op Kernel Launch:      Yes
    Device PCI Domain ID / Bus ID / location ID:   0 / 1 / 0
    Compute Mode:
    < Default (multiple host threads can use ::cudaSetDevice() with device simultaneously) >

    deviceQuery, CUDA Driver = CUDART, CUDA Driver Version = 10.2, CUDA Runtime Version = 10.2, NumDevs = 1
    Result = PASS
    \end{lstlisting}
    \begin{lstlisting}
    [CUDA Bandwidth Test] - Starting...
    Running on...

    Device 0: GeForce RTX 2060
    Quick Mode

    Host to Device Bandwidth, 1 Device(s)
    PINNED Memory Transfers
    Transfer Size (Bytes)	Bandwidth(GB/s)
    32000000			12.8

    Device to Host Bandwidth, 1 Device(s)
    PINNED Memory Transfers
    Transfer Size (Bytes)	Bandwidth(GB/s)
    32000000			13.1

    Device to Device Bandwidth, 1 Device(s)
    PINNED Memory Transfers
    Transfer Size (Bytes)	Bandwidth(GB/s)
    32000000			285.7

    Result = PASS

    NOTE: The CUDA Samples are not meant for performance measurements. Results may vary when GPU Boost is enabled.
    \end{lstlisting}

    \section{Benchmarki}
    Przyjęto następujące rozmiary problemów: 10, 100, 1024, 10240, 102400,
    1024000, 10240000, 102400000.
    W pomiarach nie są uwzględniane czasy: alokacji/dealokacji pamięci, kopiowania
    tablic \texttt{host->device} oraz  \texttt{device->host}.
    \begin{figure}[H]
        \centering
        \begin{tikzpicture}
    \begin{loglogaxis}[
        xlabel=Rozmiar problemu,
        ylabel=Czas wykonania (ms),
        width=0.7*\columnwidth,
        legend pos=north east]
        \addplot table[mark=*, x index=0, y index=1] {results/cpu_gpu_performance.dat};
        \legend{cpu}
    \end{loglogaxis}
\end{tikzpicture}

        \caption{Wydajność CPU w zależności od rozmiaru problemu.
        Wykres w skali logarytmicznej.}
        \label{fig:cpu-performance}
    \end{figure}

    Wykres~\ref{fig:cpu-performance} przedstawia wydajność CPU w zależności od rozmiaru problemu.
    Dodawanie wektorów w wersji dla CPU zostało napisane sekwencyjnie.
    Wyniki są spodziewane.
    Czas wykonania na CPU rośnie praktycznie liniowo poza pierwszymi dwoma pomiarami,
    które są na tyle małe, że różnica w czasie ich wykonania może się mieścić w granicach
    błędu statystycznego.

    \begin{figure}[H]
        \centering
        \begin{tikzpicture}
    \begin{loglogaxis}[
        xlabel=Rozmiar problemu,
        ylabel=Czas wykonania (ms),
        width=0.7*\columnwidth,
        legend pos=north east]
        \addplot table[mark=*, x index=0, y index=2] {results/cpu_gpu_performance.dat};
        \legend{gpu}
    \end{loglogaxis}
\end{tikzpicture}

        \caption{Wydajność GPU w zależności od rozmiaru problemu.
        Wykres w skali logarytmicznej.}
        \label{fig:gpu-performance}
    \end{figure}

    Wykres~\ref{fig:gpu-performance} przedstawia wydajność GPU w zależności od rozmiaru problemu.
    Spodziewałbym się, że w momencie, w którym rozmiar problemu zaczyna być większy
    od łącznej liczby rdzeni (1920), czas wykonania nieco się zmieni, ale
    możliwe, że ze względu na małe rozmiary problemu jest to niewidoczne.
    Zauważalny wzrost czasu wykonania następuje dopiero od rozmiaru problemu równego 102400.
    Benchmarki na GPU były wykonane zawsze z maksymalną liczbą wątków per blok i minimalną
    liczbą bloków, która zapewnia pokrycie całego wektora.

    \begin{figure}[H]
        \centering
        \begin{tikzpicture}
    \begin{loglogaxis}[
        xlabel=Rozmiar problemu,
        ylabel=Czas wykonania (ms),
        width=0.7*\columnwidth,
        legend pos=north west]
        \addplot table[mark=*, x index=0, y index=1] {results/cpu_gpu_performance.dat};
        \addplot table[mark=*, x index=0, y index=2] {results/cpu_gpu_performance.dat};
        \legend{cpu, gpu}
    \end{loglogaxis}
\end{tikzpicture}

        \caption{Wydajność CPU i GPU w zależności od rozmiaru problemu.
        Wykres w skali logarytmicznej.}
        \label{fig:cpu-gpu-performance}
    \end{figure}

    Wykres~\ref{fig:cpu-gpu-performance} przedstawia wcześniejsze dwa wykresy, nałożone na
    siebie w skali logarytmicznej.
    Na początku CPU jest nieco szybsze.
    Wynika to z większego taktowania procesora.
    Dla ostatniego rozmiaru problemu wydajność GPU jest dwa rzędy wielkości większa.

    \begin{figure}[H]
        \centering
        \begin{tikzpicture}
    \begin{axis}[
        ybar,
        xlabel=Liczba bloków,
        ylabel=Czas wykonania (ms),
        width=0.7*\columnwidth,
        legend pos=north east]
        \addplot table[mark=*, x index=0, y index=1] {results/gpu_block_performance.dat};
    \end{axis}
\end{tikzpicture}

        \caption{Wydajność GPU w zależności od ilości bloków.}
        \label{fig:gpu-block-performance}
    \end{figure}
    Wykres~\ref{fig:gpu-block-performance} przedstawia wydajność GPU w zależności od
    liczby bloków.
    Rozmiar problemu był równy 1024000000.
    Zostały ustalone następujące proporcje wątki/boki:
    1024/1000000, 64/16000000, 32/32000000.
    Widzimy, że najlepiej wypada konfiguracja, w której używamy najmniejszej liczby bloków
    przy maksymalnej liczbie wątków per blok.


%    \printbibliography
\end{document}
